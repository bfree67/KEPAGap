\chapter{Introduction} 

Data and information systems play a critical role in environmental compliance management due to the large types and quantities of data collected. A gap analysis was conducted on the Environmental Information Systems (EISs) used by the Kuwait Environment Public Authority (KEPA) to meet their responsibilities under the Environment Protection Law 42/2014 (EPL). The gap analysis was conducted by a consultant under the United Nations Development Program (UNDP) Kuwait Environmental Governance Initiative (KEGI) Project 00096804 beginning on 22 Nov 2018. The project was managed with KEPA through the Strategic Planning Office.

KEPA has invested heavily in advanced EISs and methods to enhance their geodatabases. Nonetheless, the project found that KEPA has significant gaps in data management and compliance applications that are required to meet their compliance obligations under the EPL and will assist in meeting their long term Sustainable Development Goals (SDGs).

\section{Project description and scope}

This project looked at EISs used by KEPA to manage and process data for internal and external consumption. As part of the project, an on-line survey was prepared and distributed to KEPA staff in order gauge use and familiarity with different EIS applications. Interviews with department heads and staff within KEPA took place to evaluate specific tools and datasets. External stakeholders were also interviewed to determine what data products they provided KEPA as well as what data products and services they would like in return.

The project evaluated existing EIS applications but acknowledges that several new applications and major enhancement to existing applications, are being developed and will be introduced in the coming year. These application were considered in the gap analysis as well.


\section{Report format}
The report is presented in the following manner:

\begin{itemize}
\item Chapter 2 provides a brief background of data processing and statistical analysis as well as descriptions of the EPL and its associated by-laws. This chapter also describes the primary EIS applications currently being used or planned within KEPA.
\item Chapter 3 describes the organization of KEPA by departments and their data generation/consumption requirements. 
\item Chapter 4 describes the different data sets generated and used within KEPA and their stakeholders, as well as reviewing different data sets and comparing to results published through the CSB.
\item Chapter 5 provides a summary of recommendations in different areas and action plan for capacity building and required resources.


\end{itemize}