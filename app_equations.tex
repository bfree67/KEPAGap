\appendix{Equations}

The general equations for Skewness, $S$, is

\begin{equation}
\label{eq:skewness}
S = \frac{\sum_{i=1}^{N}\left (x_{i}-\bar{x} \right )^{3}}{N\sigma^{3}}
\end{equation}

\noindent
and for Kurtosis, $K$, is 

\begin{equation}
\label{eq:kurtosis}
K = \frac{\sum_{i=1}^{N}\left (x_{i}-\bar{x} \right )^{4}}{N\sigma^{4}}
\end{equation}

\noindent
where $x$ is an individual data point, $\bar{x}$ is the data mean, $\sigma$ is the standard deviation, and $N$ is the number of data points \citep{Cristelli2012}.    Some Kurtosis formulas apply a correction term of ``-3" to Eq \ref{eq:kurtosis} (called excess Kurtosis) in order to bias the equation to obtain K = 0 for a normal distribution.  Kurtosis may also be calculated as shown in eq \ref{eq:xskurtosis}.

\begin{equation}
\label{eq:xskurtosis}
K = \frac{N(N+1)}{(N-1)(N-2)(N-3)} \frac{\sum_{i=1}^{N}\left (x_{i}-\bar{x} \right )^{4}}{\sigma^{4}}
\end{equation}

The first term, $(N(N+1))/((N-1)(N-2)(N-3))$, is approximately 1/N for large values of N (N$>$200), thus reducing eq \ref{eq:xskurtosis} down to eq \ref{eq:kurtosis} \citep{Cox2010}.