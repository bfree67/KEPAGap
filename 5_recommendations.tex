\chapter{Recommendations}

\section{Software recommendations}
\begin{enumerate}
\item Bring EQuIS maintenance up to date and use for all raw data collection and storage.
\item Implement an online incident/spill reporting system.
\item Adopt and utilize a common industrial code system such as the UN ISIC or NAIC for facilities processes.
\item Adopt and utilize a common facility registration system that assigns a national registration number for each stakeholder. 
\item Implement a compliance management system that tracks stakeholder registrations and permits.
\item Provide access to eMISK products to government stakeholders such as EP and MOH.
\item Implement a violation/case management system with KEPA/MOI/MOJ to share common case information.
\item Implement a coastal management system with all 6 stakeholders for sampling and reporting (EQuIS).
\item Use AQMIS for GHG inventorying instead of IPCC desktop application.
\item Install R and RStudio to provide immediate statistical analysis capabilities.
\item Upgrade AQMIS with Permit and Compliance modules.
\end{enumerate}

\section{Data processing recommendations}
\begin{enumerate}
\item Standardize email and website domain names to (@epa.kw.gov). 
\item Issue standard methods for analysing water compliance levels.
\item Issue agency procedures for handling missing data, censored data, and outliers.
\item Install atmospheric pressure sensors in air monitoring stations.
\item Issue CEMS installation requirements that include volume rate measurements.
\item Issue ground water monitoring well installation specifications and sampling plan.
\item Coordinate drinking water testing with KEPA/MEW/MOH and use common reporting platform (EQuIS).
\item Randomly sample water sites (day of week/time). Take composite samples.
\item Establish an analytical lab QA/QC program that submits spiked samples for testing.
\item Use UAS-based photogrammetry to identify coastline, desert, and nature reserve degradation and construction.
\item Require commercial labs to submit using EDDs.
\item Use Tier 3 estimation efforts for GHG calculations to check with Tier 1 results.
\item Modify IPCC emission factors for Kuwait feedstocks.
\item Require bore logs to be submitted using EDD (EQuIS).
\item Store raw data sets with EQuIS.
\item Standardize air monitoring requirements and issue conversion factors for DOAS-Chemiflourescence methods.
\item Use standardized EDDs for data submittals to EQuIS for different offices.
\item Include CAS numbers with chemical names to insure that the proper chemical is being represented.
\item Conduct regional haze study.
\item Prepare noise maps and sample ambient noise levels.
\item Standardize air dispersion models and provide common prognostic weather data.
\item Install visibility sensors in air monitoring stations.
\end{enumerate}

\begin{figure}[H]
\centering
\includegraphics[width=\linewidth,keepaspectratio]{images/recommend1.png} 
\caption{Initial recommendation for new data work flow.}
\label{fig:recommend1}
\end{figure}

\section{External stakeholder recommendations}
\begin{enumerate}
\item Communicate technical documents in English with Arabic summary.
\item Establish source registration program (but don’t call it permit – Existing Source Review/New Source Review).
\item Provide official English translations of issued regulations.
\item Establish internal metrics to monitor department performances.
\item Provide VPN/secure Read-only access to eMISK for CSB and EP using Tableau.
\end{enumerate}

\section{Prioritized recommendations}

An action plan was prepared by prioritizing the recommendations by assigning priority and cost factors to each recommendation and multiplying them to get a priority score. Priority was based on 1 for immediate, 2 for short term, and 3 for long term. Cost was based on 1 for no cost, 2 for costs less than 15,000 KD, and 3 for costs greater than 15,000 KD. The prioritized recommendations are on Table \ref{tab:priorities}.

Many of the recommendations refer back to using EQuIS for raw data submittal in order to provide a secure storage and archive for data and metadata. It also provides an easy and secure method for stakeholders to transfer datasets using email while reducing work load on analysts tasked to manually transfer data between systems.

\begin{table}[H]
\centering
\caption{Prioritized recommendations}
\label{tab:priorities}
\resizebox{\columnwidth}{!}{%
\begin{tabular}{@{}clccc@{}}
\toprule
\textbf{Item} & \textbf{Recommendation} & \textbf{Priority} & \textbf{Cost} & \textbf{Score} \\ \midrule
1 & Adopt and utilize a common industrial code system such as the UN ISIC or NAIC for facilities processes. & 1 & 1 & 1 \\
2 & Install R and RStudio to provide immediate statistical analysis capabilities. & 1 & 1 & 1 \\
3 & Issue standard methods for analysing water compliance levels. & 1 & 1 & 1 \\
4 & Issue agency procedures for handling missing data, censored data, and outliers. & 1 & 1 & 1 \\
5 & Issue CEMS installation requirements that include volume rate measurements. & 1 & 1 & 1 \\
6 & Randomly sample water sites (day of week/time). Take composite samples. & 1 & 1 & 1 \\
7 & Require commercial labs to submit using EDDs. & 1 & 1 & 1 \\
8 & Require bore logs to be submitted using EDD (EQuIS). & 1 & 1 & 1 \\
9 & Store raw data sets with EQuIS. & 1 & 1 & 1 \\
10 & Use standardized EDDs for data submittals to EQuIS for different offices. & 1 & 1 & 1 \\
11 & Include CAS numbers with chemical names to insure proper chemical is being represented. & 1 & 1 & 1 \\
12 & Bring EQuIS maintenance up to date and use for all raw data collection and storage. & 1 & 2 & 2 \\
13 & Issue ground water monitoring well installation specifications and sampling plan. & 2 & 1 & 2 \\
14 & Coordinate drinking water testing with KEPA/MEW/MOH and use common reporting platform (EQuIS). & 2 & 1 & 2 \\
15 & Conduct regional haze study & 2 & 1 & 2 \\
16 & Standardize air dispersion modeling and provide common prognostic weather data for models. & 2 & 1 & 2 \\
17 & Implement an online incident/spill reporting system. & 1 & 2 & 2 \\
18 & Install visibility sensors in air monitoring stations & 1 & 3 & 3 \\
19 & Establish internal metrics to monitor department performances. & 3 & 1 & 3 \\
20 & Use AQMIS for GHG inventorying instead of IPCC desktop application. & 2 & 2 & 4 \\
21 & Communicate technical documents in English with Arabic summary. & 2 & 2 & 4 \\
22 & Provide official English translations of issued regulations. & 2 & 2 & 4 \\
23 & Upgrade AQMIS with Permit and Compliance modules & 2 & 2 & 4 \\
24 & Use UAS-based photogrammetry to identify coastline, desert, and nature reserve degradation and construction. & 2 & 2 & 4 \\
25 & Provide access to eMISK products to government stakeholders such as EP and MOH. & 2 & 3 & 6 \\
26 & Implement a coastal management system with all 6 stakeholders for sampling and reporting (EQuIS). & 2 & 3 & 6 \\
27 & Establish an analytical lab QA/QC program that submits spiked samples for testing. & 3 & 2 & 6 \\
28 & Use Tier 3 estimation efforts for GHG calculations to check with Tier 1 results. & 3 & 2 & 6 \\
29 & Standardize air monitoring requirements and issue conversion factors for DOAS-Chemiflourescence methods. & 3 & 2 & 6 \\
30 & Establish source registration program (but don’t call it permit – Existing Source Review/New Source Review). & 3 & 2 & 6 \\
31 & Adopt and utilize a common facility registration system that assigns a national registration number for each stakeholder. & 3 & 3 & 9 \\
32 & Implement a compliance management system that tracks stakeholder registrations and permits. & 3 & 3 & 9 \\
33 & Implement a violation/case management system with KEPA/MOI/MOJ to share common case information. & 3 & 3 & 9 \\
34 & Standardize email and website domain names to (@epa.kw.gov). & 3 & 3 & 9 \\
35 & Install atmospheric pressure sensors in air monitoring stations. & 3 & 3 & 9 \\
36 & Modify IPCC emission factors for Kuwait feedstocks. & 3 & 3 & 9 \\
37 & Prepare noise maps and sample ambient noise levels. & 3 & 3 & 9 \\
38 & Provide VPN/secure Read-only access to eMISK for CSB and EP using Tableau. & 3 & 3 & 9 \\ \bottomrule
\end{tabular}
} %end resize
\end{table}

\section{Roadmap for action}
Evaluating the recommendations in Table \ref{tab:priorities} show that there are several common themes, namely applying the EQuIS data management system,  issuing key standards and documentation, and enhancing data categories and statistical analysis. The individual recommendations are extracted from Table \ref{tab:priorities} and presented in subsets. Recommendations associated with EQuIS are shown in Table \ref{tab:recequis}. Recommendations associated with issuing clear documentation is shown in Table \ref{tab:recequis}. Recommendations associated with data series and analysis are shown in Table \ref{tab:recdata}.

\begin{table}[H]
\centering
\caption{Recommendations based on EQuIS.}
\label{tab:recequis}
\resizebox{\columnwidth}{!}{%
\begin{tabular}{@{}ll@{}}
\toprule
\textbf{Item} & \textbf{Recommendation} \\ \midrule
7 & Require commercial labs to submit using EDDs. \\
8 & Require bore logs to be submitted using EDD (EQuIS). \\
9 & Store raw data sets with EQuIS. \\
10 & Use standardized EDDs for data submittals to EQuIS for different offices. \\
12 & Bring EQuIS maintenance up to date and use for all raw data collection and storage. \\
14 & Coordinate drinking water testing with KEPA/MEW/MOH and use common reporting platform (EQuIS). \\
17 & Implement an online incident/spill reporting system. \\
25 & Implement a coastal management system with all 6 stakeholders for sampling and reporting (EQuIS). \\
26 & Establish an analytical lab QA/QC program that submits spiked samples for testing. \\ \bottomrule
\end{tabular}
} % end resize
\end{table}


\begin{table}[H]
\centering
\caption{Recommendations based on documentation.}
\label{tab:recdocuments}
\resizebox{\columnwidth}{!}{%
\begin{tabular}{@{}ll@{}}
\toprule
\textbf{Item} & \textbf{Recommendation} \\ \midrule
3 & Issue standard methods for analysing water compliance levels. \\
4 & Issue agency procedures for handling missing data, censored data, and outliers. \\
5 & Issue CEMS installation requirements that include volume rate measurements. \\
13 & Issue ground water monitoring well installation specifications and sampling plan. \\
20 & Communicate technical documents in English with Arabic summary. \\
21 & Provide official English translations of issued regulations. \\
26 & Establish an analytical lab QA/QC program that submits spiked samples for testing. \\
30 & Adopt and utilize a common facility registration system that assigns a national registration number for each stakeholder. \\ \bottomrule
\end{tabular}
} % end resize
\end{table}


\begin{table}[H]
\centering
\caption{Recommendations based on data analysis.}
\label{tab:recdata}
\resizebox{\columnwidth}{!}{%
\begin{tabular}{@{}ll@{}}
\toprule
\textbf{Item} & \textbf{Recommendation} \\ \midrule
1 & Adopt and utilize a common industrial code system such as the UN ISIC or NAIC for facilities processes. \\
2 & Install R and RStudio to provide immediate statistical analysis capabilities. \\
6 & Randomly sample water sites (day of week/time). Take composite samples. \\
11 & Include CAS numbers with chemical names to insure proper chemical is being represented. \\
35 & Modify IPCC emission factors for Kuwait feedstocks. \\ \bottomrule
\end{tabular}
} % end resize
\end{table}

Based on this clustering of recommendations, the following course of action should be considered by KEPA to close the most EIS gaps in the shortest period of time and with the least amount of resources:

\begin{enumerate}
\item Update EQuIS maintenance.
\item Issue standard methods to sample, analyze and report environmental chemistry using the KEDD.
\item Issue internal procedures to handle data, assign industrial sector codes, and use common facility registration codes.
\item Train internal and external stakeholders on preparing KEDDs and using the EDP for submittal checking.
\item Install R \& RStudio and provide training for each department based on individual section requirements.
\item Use the KEDD spill reporting and waste reporting capabilities until an upgraded system is available.
\item Upgrade the AQMIS permit and compliance module to provide compliance management capabilities.
\end{enumerate}

\section{Applicability to Sustainable Development Goals}

Recommendations listed in Table \ref{tab:priorities} can directly support Kuwait's responsibilities unde the UN's Sustainable Development Goals (SDGs).  The top 10 recommendations, based on recommended implementation priority, is shown in Table \ref{tab:sdgtop10}.

\begin{table}[H]
\centering
\caption{Top 10 recommendations to support SDGs.}
\label{tab:sdgtop10}
\resizebox{\columnwidth}{!}{%
\begin{tabular}{@{}cl@{}}
\toprule
\textbf{Priority} & \textbf{Recommendation} \\ \midrule
1 & Install R and RStudio to provide immediate statistical analysis capabilities. \\
2 & Issue agency procedures for handling missing data, censored data, and outliers. \\
3 & Store raw data sets with EQuIS. \\
4 & Use standardized EDDs for data submittals to EQuIS for different offices. \\
5 & Communicate technical documents in English with Arabic summary. \\
6 & Provide official English translations of issued regulations. \\
7 & Adopt and utilize a common industrial code system such as the UN ISIC or NAIC for facilities processes. \\
8 & Require commercial labs to submit using EDDs. \\
9 & Bring EQuIS maintenance up to date and use for all raw data collection and storage. \\
10 & Implement a compliance management system that tracks stakeholder registrations and permits. \\ \bottomrule
\end{tabular}
} %end resize
\end{table}

A detailed description of the applicable SDGs and how the prioritized recommendations in Table \ref{tab:sdgtop10} is shown in Appendix D.