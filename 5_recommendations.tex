\chapter{Recommendations}

\section{Software recommendations}
\begin{enumerate}
\item Bring EQuIS maintenance up to date and use for all raw data collection and storage.
\item Implement an online incident/spill reporting system.
\item Adopt and utilize a common industrial code system such as the UN ISIC or NAIC for facilities processes.
\item Adopt and utilize a common facility registration system that assigns a national registration number for each stakeholder. 
\item Implement a compliance management system that tracks stakeholder registrations and permits.
\item Provide access to eMISK products to government stakeholders such as EP and MOH.
\item Implement a violation/case management system with KEPA/MOI/MOJ to share common case information.
\item Implement a coastal management system with all 6 stakeholders for sampling and reporting (EQuIS).
\item Use AQMIS for GHG inventorying instead of IPCC desktop application.
\item Install R and RStudio to provide immediate statistical analysis capabilities.
\item Upgrade AQMIS with Permit and Compliance modules.
\end{enumerate}

\section{Data processing recommendations}
\begin{enumerate}
\item Standardize email and website domain names to (@epa.kw.gov). 
\item Issue standard methods for analysing water compliance levels.
\item Issue agency procedures for handling missing data, censored data, and outliers.
\item Install atmospheric pressure sensors in air monitoring stations.
\item Issue CEMS installation requirements that include volume rate measurements.
\item Issue ground water monitoring well installation specifications and sampling plan.
\item Coordinate drinking water testing with KEPA/MEW/MOH and use common reporting platform (EQuIS).
\item Randomly sample water sites (day of week/time). Take composite samples.
\item Establish an analytical lab QA/QC program that submits spiked samples for testing.
\item Use UAS-based photogrammetry to identify coastline, desert, and nature reserve degradation and construction.
\item Require commercial labs to submit using EDDs.
\item Use Tier 3 estimation efforts for GHG calculations to check with Tier 1 results.
\item Modify IPCC emission factors for Kuwait feedstocks.
\item Require bore logs to be submitted using EDD (EQuIS).
\item Store raw data sets with EQuIS.
\item Standardize air monitoring requirements and issue conversion factors for DOAS-Chemiflourescence methods.
\item Use standardized EDDs for data submittals to EQuIS for different offices.
\item Include CAS numbers with chemical names to insure proper chemical is being represented.
\item Conduct regional haze study.
\item Prepare noise maps and sample ambient noise levels.
\item Standardize air dispersion models and provide common prognostic weather data.
\end{enumerate}

\begin{figure}[H]
\centering
\includegraphics[width=\linewidth,keepaspectratio]{images/recommend1.png} 
\caption{Initial recommendation for new data work flow.}
\label{fig:recommend1}
\end{figure}

\section{External stakeholder recommendations}
\begin{enumerate}
\item Communicate technical documents in English with Arabic summary.
\item Establish source registration program (but don’t call it permit – Existing Source Review/New Source Review).
\item Provide official English translations of issued regulations.
\item Establish internal metrics to monitor department performances.
\item Provide VPN/secure Read-only access to eMISK for CSB and EP using Tableau.
\end{enumerate}

\section{Action Plan}

An action plan was prepared by prioritizing the recommendations by assigning priority and cost factors to each recommendation and multiplying them to get a priority score. Priority was based on 1 for immediate, 2 for short term, and 3 for long term. Cost was based on 1 for no cost, 2 for costs less than 15,000 KD, and 3 for costs greater than 15,000 KD. The prioritized recommendations are on Table \ref{tab:priorities}.

Many of the recommendations refer back to using EQuIS for raw data submittal in order to provide a secure storage and archive for data and metadata. It also provides an easy and secure method for stakeholders to transfer datasets using email while reducing work load on analysts tasked to manually transfer data between systems.

\begin{table}[H]
\centering
\caption{Prioritized recommendations}
\label{tab:priorities}
\resizebox{\columnwidth}{!}{%
\begin{tabular}{@{}clccc@{}}
\toprule
\textbf{Item} & \textbf{Recommendation} & \textbf{Priority} & \textbf{Cost} & \textbf{Score} \\ \midrule
1 & Adopt and utilize a common industrial code system such as the UN ISIC or NAIC for facilities processes. & 1 & 1 & 1 \\
2 & Install R and RStudio to provide immediate statistical analysis capabilities. & 1 & 1 & 1 \\
3 & Issue standard methods for analysing water compliance levels. & 1 & 1 & 1 \\
4 & Issue agency procedures for handling missing data, censored data, and outliers. & 1 & 1 & 1 \\
5 & Issue CEMS installation requirements that include volume rate measurements. & 1 & 1 & 1 \\
6 & Randomly sample water sites (day of week/time). Take composite samples. & 1 & 1 & 1 \\
7 & Require commercial labs to submit using EDDs. & 1 & 1 & 1 \\
8 & Require bore logs to be submitted using EDD (EQuIS). & 1 & 1 & 1 \\
9 & Store raw data sets with EQuIS. & 1 & 1 & 1 \\
10 & Use standardized EDDs for data submittals to EQuIS for different offices. & 1 & 1 & 1 \\
11 & Include CAS numbers with chemical names to insure proper chemical is being represented. & 1 & 1 & 1 \\
12 & Bring EQuIS maintenance up to date and use for all raw data collection and storage. & 1 & 2 & 2 \\
13 & Issue ground water monitoring well installation specifications and sampling plan. & 2 & 1 & 2 \\
14 & Coordinate drinking water testing with KEPA/MEW/MOH and use common reporting platform (EQuIS). & 2 & 1 & 2 \\
15 & Conduct regional haze study & 2 & 1 & 2 \\
16 & Standardize air dispersion modeling and provide common prognostic weather data for models. & 2 & 1 & 2 \\
17 & Implement an online incident/spill reporting system. & 1 & 2 & 2 \\
18 & Establish internal metrics to monitor department performances. & 3 & 1 & 3 \\
19 & Use AQMIS for GHG inventorying instead of IPCC desktop application. & 2 & 2 & 4 \\
20 & Communicate technical documents in English with Arabic summary. & 2 & 2 & 4 \\
21 & Provide official English translations of issued regulations. & 2 & 2 & 4 \\
22 & Upgrade AQMIS with Permit and Compliance modules & 2 & 2 & 4 \\
23 & Use UAS-based photogrammetry to identify coastline, desert, and nature reserve degradation and construction. & 2 & 2 & 4 \\
24 & Provide access to eMISK products to government stakeholders such as EP and MOH. & 2 & 3 & 6 \\
25 & Implement a coastal management system with all 6 stakeholders for sampling and reporting (EQuIS). & 2 & 3 & 6 \\
26 & Establish an analytical lab QA/QC program that submits spiked samples for testing. & 3 & 2 & 6 \\
27 & Use Tier 3 estimation efforts for GHG calculations to check with Tier 1 results. & 3 & 2 & 6 \\
28 & Standardize air monitoring requirements and issue conversion factors for DOAS-Chemiflourescence methods. & 3 & 2 & 6 \\
29 & Establish source registration program (but don’t call it permit – Existing Source Review/New Source Review). & 3 & 2 & 6 \\
30 & Adopt and utilize a common facility registration system that assigns a national registration number for each stakeholder. & 3 & 3 & 9 \\
31 & Implement a compliance management system that tracks stakeholder registrations and permits. & 3 & 3 & 9 \\
32 & Implement a violation/case management system with KEPA/MOI/MOJ to share common case information. & 3 & 3 & 9 \\
33 & Standardize email and website domain names to (@epa.kw.gov). & 3 & 3 & 9 \\
34 & Install atmospheric pressure sensors in air monitoring stations. & 3 & 3 & 9 \\
35 & Modify IPCC emission factors for Kuwait feedstocks. & 3 & 3 & 9 \\
36 & Prepare noise maps and sample ambient noise levels. & 3 & 3 & 9 \\
37 & Provide VPN/secure Read-only access to eMISK for CSB and EP using Tableau. & 3 & 3 & 9 \\ \bottomrule
\end{tabular}
} %end resize
\end{table}