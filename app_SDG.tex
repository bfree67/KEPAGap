\chapter{Applicability to Sustainable Development Goals}

\begin{flushleft}
In 2015, the UN adopted 17 Sustainable Development Goals (SDGs) as continuation of the Millennium Development Goals that expired in the same year \citep{SDG2015}. Within those goals, six goals are directly linked to environmental management and were assigned to KEPA by CSB as the primary custodian. The list of SDGs is given in Table \ref{tab:SDG}. Each SDG has a series of targets and associated indicators that can be used to track national progress to each goal.
\end{flushleft}

\begin{table}[H]
\centering
\caption{SDGs assigned to KEPA}
\label{tab:SDG}
\begin{tabular}{@{}ll@{}}
\toprule
\textbf{SDG \#} & \textbf{Description}               \\ \midrule
SDG 6           & Clean Water and Sanitation         \\
SDG 7           & Affordable and Clean Energy        \\
SDG 11          & Sustainable Cities and Communities \\
SDG 13          & Climate Change                     \\
SDG 14          & Life Below Water                   \\
SDG 15          & Life on Land                       \\ \bottomrule
\end{tabular}
\end{table}

\begin{flushleft}
Most of the recommendations prioritized in Table \ref{tab:priorities} have direct bearing on the ability to properly measure the indicators established for the SDGs. Each recommendation was evaluated against the six SDGs in Table \ref{tab:SDG} to determine if it could directly impact the data collection process and analysis required to generate the indicators. Some of the recommendations were internal to KEPA and had no impact to SDG indicators. These recommendations were considered not applicable.
\end{flushleft}

\begin{table}[H]
\centering
\caption{Recommendations associated with SDGs}
\label{tab:recsdg}
\resizebox{\columnwidth}{!}{%
\begin{tabular}{@{}clcccccc@{}}
\toprule
\textbf{Item} & \textbf{Recommendation} & \textbf{SDG 6} & \textbf{SDG 7} & \textbf{SDG 11} & \textbf{SDG 13} & \textbf{SDG 14} & \textbf{SDG 15} \\ \midrule
1 & Adopt and utilize a common industrial code system such as the UN ISIC or NAIC for facilities processes. & X & X & X & X &  & X \\
2 & Install R and RStudio to provide immediate statistical analysis capabilities. & X & X & X & X & X & X \\
3 & Issue standard methods for analysing water compliance levels. & X &  &  &  & X &  \\
4 & Issue agency procedures for handling missing data, censored data, and outliers. & X & X & X & X & X & X \\
5 & Issue CEMS installation requirements that include volume rate measurements. &  & X & X & X &  &  \\
6 & Randomly sample water sites (day of week/time). Take composite samples. & X &  &  &  & X &  \\
7 & Require commercial labs to submit using EDDs. & X & X & X &  & X & X \\
8 & Require bore logs to be submitted using EDD (EQuIS). & X & X & X &  &  & X \\
9 & Store raw data sets with EQuIS. & X & X & X & X & X & X \\
10 & Use standardized EDDs for data submittals to EQuIS for different offices. & X & X & X & X & X & X \\
11 & Include CAS numbers with chemical names to insure proper chemical is being represented. & X &  & X &  & X & X \\
12 & Bring EQuIS maintenance up to date and use for all raw data collection and storage. & X & X & X &  & X & X \\
13 & Issue ground water monitoring well installation specifications and sampling plan. & X & X &  &  &  &  \\
14 & Coordinate drinking water testing with KEPA/MEW/MOH and use common reporting platform (EQuIS). & X &  & X &  &  &  \\
15 & Conduct regional haze study &  &  & X & X &  &  \\
16 & Standardize air dispersion modeling and provide common prognostic weather data. &  & X & X &  &  &  \\
17 & Implement an online incident/spill reporting system. &  & X &  &  & X & X \\
18 & Install visibility sensors in air monitoring stations. &  &  & X &  &  &  \\
19 & Establish internal metrics to monitor department performances. & \multicolumn{6}{c}{Not Applicable} \\
20 & Use AQMIS for GHG inventorying instead of IPCC desktop application. &  &  & X & X &  &  \\
21 & Communicate technical documents in English with Arabic summary. & X & X & X & X & X & X \\
22 & Provide official English translations of issued regulations. & X & X & X & X & X & X \\
23 & Upgrade AQMIS with Permit and Compliance modules. &  & X & X &  &  &  \\
24 & Use UAS-based photogrammetry to identify coastline, desert, and nature reserve degradation and construction. &  &  & X &  &  &  \\
25 & Provide access to eMISK products to government stakeholders such as EP and MOH. & X &  &  &  &  &  \\
26 & Implement a coastal management system with all 6 stakeholders for sampling and reporting (EQuIS). &  &  & X &  & X &  \\
27 & Establish an analytical lab QA/QC program that submits spiked samples for testing. & X &  &  &  & X & X \\
28 & Use Tier 3 estimation efforts for GHG calculations to check with Tier 1 results. &  &  &  & X &  &  \\
29 & Standardize air monitoring requirements and issue conversion factors for DOAS-Chemiflourescence methods. &  &  & X & X &  &  \\
30 & Establish source registration program (but don’t call it permit – Existing Source Review/New Source Review). &  & X & X & X &  & X \\
31 & Adopt and utilize a common facility registration system that assigns a national registration number for each stakeholder. &  & X & X & X &  & X \\
32 & Implement a compliance management system that tracks stakeholder registrations and permits. & X & X & X & X & X &  \\
33 & Implement a violation/case management system with KEPA/MOI/MOJ to share common case information. &  & X & X &  & X &  \\
34 & Standardize email and website domain names to (@epa.kw.gov). & \multicolumn{6}{c}{Not applicable} \\
35 & Install atmospheric pressure sensors in air monitoring stations. &  &  & X & X &  &  \\
36 & Modify IPCC emission factors for Kuwait feedstocks. &  &  &  & X &  &  \\
37 & Prepare noise maps and sample ambient noise levels. &  & X & X &  &  &  \\
38 & Provide VPN/secure Read-only access to eMISK for CSB and EP using Tableau. & \multicolumn{6}{c}{Not applicable} \\ \bottomrule
\end{tabular}
} %end resize
\end{table}

\begin{flushleft}
From the results in Table \ref{tab:recsdg}, the recommendations with the most applicability to the SDGS can be extracted. A list of the top 10 recommendations based on SDG impacts is shown in Table \ref{tab:sdgimpact}.
\end{flushleft}

\begin{table}[H]
\centering
\caption{Top 10 recommendations based on SDG impacts.}
\label{tab:sdgimpact}
\resizebox{\columnwidth}{!}{%
\begin{tabular}{@{}cl@{}}
\toprule
\textbf{Item} & \textbf{Recommendation} \\ \midrule
2 & Install R and RStudio to provide immediate statistical analysis capabilities. \\
4 & Issue agency procedures for handling missing data, censored data, and outliers. \\
9 & Store raw data sets with EQuIS. \\
10 & Use standardized EDDs for data submittals to EQuIS for different offices. \\
21 & Communicate technical documents in English with Arabic summary. \\
22 & Provide official English translations of issued regulations. \\
1 & Adopt and utilize a common industrial code system such as the UN ISIC or NAIC for facilities processes. \\
7 & Require commercial labs to submit using EDDs. \\
12 & Bring EQuIS maintenance up to date and use for all raw data collection and storage. \\
32 & Implement a compliance management system that tracks stakeholder registrations and permits. \\ \bottomrule
\end{tabular}
} %end resize
\end{table}